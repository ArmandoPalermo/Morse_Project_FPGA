\documentclass[9pt]{beamer}
% Created By Gouthaman KG
%~~~~~~~~~~~~~~~~~~~~~~~~~~~~~~~~~~~~~~~~~~~~~~~~~~~~~~~~~~~~~~~~~~~~~~~~~~~~~~
% Use roboto Font (recommended)
\usepackage[sfdefault]{roboto}
\usepackage[utf8]{inputenc}
\usepackage[T1]{fontenc}
%~~~~~~~~~~~~~~~~~~~~~~~~~~~~~~~~~~~~~~~~~~~~~~~~~~~~~~~~~~~~~~~~~~~~~~~~~~~~~~

%~~~~~~~~~~~~~~~~~~~~~~~~~~~~~~~~~~~~~~~~~~~~~~~~~~~~~~~~~~~~~~~~~~~~~~~~~~~~~~
% Define where theme files are located. ('/styles')
\usepackage{styles/fluxmacros}
\usefolder{styles}
% Use Flux theme v0.1 beta
% Available style: asphalt, blue, red, green, gray 
\usetheme[style=asphalt]{flux}
%~~~~~~~~~~~~~~~~~~~~~~~~~~~~~~~~~~~~~~~~~~~~~~~~~~~~~~~~~~~~~~~~~~~~~~~~~~~~~~

%~~~~~~~~~~~~~~~~~~~~~~~~~~~~~~~~~~~~~~~~~~~~~~~~~~~~~~~~~~~~~~~~~~~~~~~~~~~~~~
% Extra packages for the demo:
\usepackage{booktabs}
\usepackage{colortbl}
\usepackage{ragged2e}
\usepackage{schemabloc}
\usepackage{hyperref}
\usebackgroundtemplate{
\includegraphics[width=\paperwidth,height=\paperheight]{assets/background.jpg}}%change this to your preferred background for the presentation.
%~~~~~~~~~~~~~~~~~~~~~~~~~~~~~~~~~~~~~~~~~~~~~~~~~~~~~~~~~~~~~~~~~~~~~~~~~~~~~~

%~~~~~~~~~~~~~~~~~~~~~~~~~~~~~~~~~~~~~~~~~~~~~~~~~~~~~~~~~~~~~~~~~~~~~~~~~~~~~~
% Informations

\title{Sistema Morse su FPGA: \\
progetto di codifica e decodifica}


\author{Colasanti Giulia, Palermo Armando }
\institute{}
\titlegraphic{Logo_unipg.png} %change this to your preferred logo or image(the image is located on the top right corner).
%~~~~~~~~~~~~~~~~~~~~~~~~~~~~~~~~~~~~~~~~~~~~~~~~~~~~~~~~~~~~~~~~~~~~~~~~~~~~~~

\begin{document}

\begin{frame} 
\centering 

\begin{figure}[H] 
\centering 
\includegraphics[width=0.2\linewidth]{Logo_unipg.png}
\end{figure} 
\textbf{\Huge SISTEMA MORSE SU FPGA: \\ 
}
\huge PROGETTO DI CODIFICA E DECODIFCA \\
\vspace{0.5cm}
\small \textbf{INGEGNERIA INFORMATICA E ROBOTICA} \\
\small SISTEMI ELETTRONICI EMBEDDED \\
\vspace{0.5cm}
\tiny A.A 2024/2025 
\end{frame}
% Generate title page


\begin{frame}

 \frametitle{TABLE OF CONTENTS}
 \tableofcontents
\end{frame}

\section{Introduction} %the content in the section will be displayed in the table of contents
\begin{frame}{Introduction: Morse code}%the content in the frame will be displayed as the title of the page
\begin{columns}
\column{0.45\textwidth} 
\begin{figure}
           \includegraphics[width=0.8\linewidth]{telegrrafo.jpg}
       \end{figure}
\column{0.55\textwidth} 
Developed in the 19th century by Samuel Morse and Alfred Vail. It was mainly used in the \textbf{telegraph} for long-distance communications.
\end{columns}
\vspace{2mm}

\begin{columns}
\column{0.65\textwidth} 
Purposes of use: 
\begin{itemize} 
\item Military field 
\item Maritime field 
\item Today it is useful in specific contexts such as the medical sector 
\end{itemize}
\column{0.35\textwidth} 
\begin{figure}
           \includegraphics[width=1\linewidth]{morse-code.jpg}
\end{figure}
\end{columns}

    
\end{frame}

\section{Project goal}
\begin{frame}{Project goal}
Implementation of a Morse code encoding and decoding system using an \textbf{FPGA Artix-7 by Xilinx}. The resources provided by the \textbf{Basys3} board will be used as input/output.
The Hardware is programmed using \textbf{Vivado} Software.
\begin{columns}
\column{0.5\textwidth} 
\begin{figure}
    \centering
    \includegraphics[width=\linewidth]{FPGABasys.png}
    \label{fig:enter-label}
\end{figure}
\column{0.5\textwidth} 
\begin{figure}
    \centering
    \includegraphics[width=0.7\linewidth]{sw.png}
    \label{fig:enter-label}
\end{figure}
\end{columns}
\end{frame}


\begin{frame}{Project goal}

\begin{figure}
           \centering
           
           \includegraphics[width=\linewidth]{FPGASchema.png}
\end{figure}

\end{frame}





\section{Project Structure}

\begin{frame}{Project Structure- RTL}
    \textbf{Register Transfer Level}: 
    \begin{itemize}
        \item How data moves between registers;
        \item What logical or arithmetic operations are performed on the data during transfers;
    \end{itemize}
    

    \begin{figure}
           \centering
           
           \includegraphics[width=\linewidth]{RTL.png}
\end{figure}
\end{frame}

\begin{frame}{Project Structure - Encoding }
Two modules:
\begin{itemize}
    \item \textbf{Mapping$\_$SW$\_$Morse}, It converts the binary input provided through the switches into an identifying Morse code encoding.
    \begin{center}
        \emph{e.g. \textcolor{red}{D} → 01000100 → 1101010000000000 →   \textcolor{red}{- . .}}
    \end{center}
    
    \item \textbf{Encoder$\_$LED},It makes the LED blink based on the value obtained from the mapping module mentioned above.
    \begin{figure}
           \centering
           
           \includegraphics[width=0.9\linewidth]{exBlink.png}
    \end{figure}
\end{itemize}


\end{frame}

\begin{frame}{Project Structure -Encoder$\_$LED Code}
    \begin{figure}
           \centering
           
           \includegraphics[width=0.6\linewidth]{ENCODERcode.png}
    \end{figure}
\end{frame}

\begin{frame}{Project Structure - Decoding}
    Three modules:
    \begin{itemize}
        \item \textbf{DebounceControlModule}, manages the unwanted mechanical bounce of the pushbuttons;
        \item \textbf{Decoder\_Morse}, interpret the input signals from the pushbuttons and convert them into binary strings (encoding as seen in the previous slides);
        \item \textbf{Segment\_Display}, performs the mapping from the Morse code entered using the pushbuttons into a binary string for turning on the 7-segment display.
    \end{itemize}
\end{frame}



\begin{frame}{Project Structure -  DebounceControlModule code}
    \begin{figure}
           \centering
           
           \includegraphics[width=\linewidth]{debounce codice .png}
    \end{figure}
\end{frame}

\begin{frame}{Project Structure -  DebounceControlModule Simulation}
    \begin{center}
        \Huge Simulated Debounce $\rightarrow$ 1 us
    \end{center}
    \begin{figure}
           \centering
           
           \includegraphics[width=\linewidth]{simulazDeboun.png}
    \end{figure}
\end{frame}

\begin{frame}{Project Structure - Flow Diagram Decoder\_Morse }
    \begin{figure}
           \centering
           
           \includegraphics[width=0.41\linewidth]{fotolflowgius.png}
    \end{figure}
\end{frame}




\begin{frame}{Project Structure - Clock Division}
    \begin{center}
        \Huge 100 MHz $\rightarrow$ 10 MHz
    \end{center}
    \begin{figure}
           \centering
           
           \includegraphics[width=\linewidth]{ClockDivision.png}
    \end{figure}
    \begin{figure}
           \centering
           
           \includegraphics[width=\linewidth]{powerresults.png}
    \end{figure}
\end{frame}

\section{Conclusions}
\begin{frame}{Conclusions and Future Developments}
    Possible improvements:
    \begin{itemize}
        \item Development of UART connection for the conversion of multiple characters simultaneously..
        \item Clock gating to prevent the unused mode from consuming power.
    \end{itemize}
\end{frame}



\begin{frame}{}
   \centering
   \Huge THANK YOU FOR YOUR ATTENTION
\centering
\end{frame}
\end{document}